\documentclass[aspectratio=169,11pt,hyperref={colorlinks=true}]{beamer}
% https://github.com/zr-tex8r/BXcjkjatype/blob/master/README-ja.md
\usepackage[whole]{bxcjkjatype}
\usetheme{boxes}
\setbeamertemplate{navigation symbols}{}
\definecolor{suse}{RGB}{2, 211, 95}
\definecolor{susedark}{RGB}{13, 44, 64}
\setbeamercolor{titlelike}{fg=suse}
\setbeamercolor{structure}{fg=suse}
\hypersetup{colorlinks,urlcolor=suse}
\setbeamertemplate{footline}[frame number]
% Inserting graphics
\usepackage{graphicx}
% Side-by-side figures, etc
\usepackage{subfigure}
% Code snippits
\usepackage{listings}
% Color stuff
\usepackage{color}
% underline
\usepackage{soul}

\usepackage{amsmath}
\usepackage{tikz}
\newcommand\RBox[1]{%
  \tikz\node[draw,rounded corners,align=center,] {#1};%
}
\usepackage{hyperref}
%\usecolortheme{buzz}
%\usecolortheme{wolverine}
%\usetheme{Boadilla}
\usepackage[T1]{fontenc}
%\usepackage{fontspec}
%\usepackage[expert, deluxe]{otf}

\definecolor{mygreen}{rgb}{0,0.6,0}
\definecolor{mygray}{rgb}{0.5,0.5,0.5}
\definecolor{mymauve}{rgb}{0.58,0,0.82}


%\usepackage{CJK}
%\pdfmapline{=genshingothic@Unicode@ <genshingothic.ttf}
% bxcjkjatype
%\setgothicfont[<ID>]{<フォントファイル名>}
%\setgothicfont{/Users/foo/Library/Fonts/genshingothic.ttf}
%\setgothicfont{/Users/foo/Library/Fonts/NotoSansCJKjp-Regular.otf}
%\setgothicfont{/Users/foo/Downloads/genshingothic-20150607/GenShinGothic-P-Normal.ttf}
%\setgothicfont{/Users/foo/Downloads/genshingothic-20150607/GenShinGothic-Regular.ttf}
%\setgothicfont{hiragino.ttc}
\setgothicfont{mplus-1p-regular.ttf}
\setCJKfamilydefault{\gtdefault}
%\setCJKfamilydefault{\mcdefault}
%\CJKforce{abcdefghijklmnopqrstuvwxyzABCDEFGHIJKLMNOPQRSTUVWXYZ}


\lstset{%
  backgroundcolor=\color{susedark},   % choose the background color; you must add \usepackage{color} or \usepackage{xcolor}
  breakatwhitespace=false,         % sets if automatic breaks should only happen at whitespace
  breaklines=true,                 % sets automatic line breaking
  captionpos=b,                    % sets the caption-position to bottom
  commentstyle=\color{suse},  % comment style
  extendedchars=true,              % lets you use non-ASCII characters; for 8-bits encodings only, does not work with UTF-8
  keepspaces=true,                 % keeps spaces in text, useful for keeping indentation of code (possibly needs columns=flexible)
  keywordstyle=\color{blue},       % keyword style
%  otherkeywords={*,...},           % if you want to add more keywords to the set
  numbersep=5pt,                   % how far the line-numbers are from the code
  numberstyle=\tiny\color{mygray}, % the style that is used for the line-numbers
  rulecolor=\color{white},         % if not set, the frame-color may be changed on line-breaks within not-black text (e.g. comments (green here))
  showspaces=false,                % show spaces everywhere adding particular underscores; it overrides 'showstringspaces'
  showstringspaces=false,          % underline spaces within strings only
  showtabs=false,                  % show tabs within strings adding particular underscores
  stringstyle=\color{suse},   % string literal style
}

\setbeamerfont{caption}{series=\normalfont,size=\fontsize{6}{8}}
%\setbeamerfont{caption}{series=\normalfont,size=\large}
\setbeamertemplate{caption}{\raggedright\insertcaption\par}

\setlength{\abovecaptionskip}{0pt}
\setlength{\floatsep}{0pt}

\author[Masayuki Igawa]{%
  \texorpdfstring{%
    \centering
    Masayuki Igawa\\
    \href{mailto:masayuki@igawa.io}{masayuki@igawa.io}\\
    \texttt{masayukig on \href{http://freenode.net/}{Freenode},
     \href{https://twitter.com/masayukig}{Twitter},
     \href{https://github.com/masayukig}{GitHub}}
  }
  {Masayuki Igawa}
}
\date{20 July, 2017}
\def\place#1{\def\@place{#1}}
\place{\href{https://ospn.connpass.com/event/60330/}{@OpenStack Days Tokyo 2017}}

\title[Non-Native-English-Speaker
\hspace{2em}\insertframenumber/\inserttotalframenumber]{OpenStackコミュニティにおける\\非英語ネイティブ話者の苦悩と奮闘記}

\setbeamercolor{background canvas}{bg=susedark}
\setbeamercolor{titlelike}{fg=white}
\setbeamercolor{structure}{fg=white}
\setbeamercolor{normal text}{fg=white}
\begin{document}

{%
% \setbeamertemplate{background canvas}{\includegraphics[width=\paperwidth,height=\paperheight]{background_title.png}}
\setbeamertemplate{footline}{}
\setbeamercolor{background canvas}{bg=susedark}
\begin{frame}[noframenumbering]
  \hypersetup{colorlinks,urlcolor=suse}
  \setbeamercolor{author}{fg=white}
  \setbeamercolor{date}{fg=white}
  \setbeamercolor{place}{fg=white}
  \titlepage{}
  \centering
  \@place \par
  \href{https://github.com/masayukig/osdt-2017-non-native-english-speaker}{github.com/masayukig/osdt-2017-non-native-english-speaker}\\
  \vspace{1em}
  \begin{flushright}
    \tiny\href{https://creativecommons.org/licenses/by/4.0/}{This work
      is licensed under a Creative Commons Attribution 4.0
      International License.}\includegraphics[scale=0.3]{cc_by.png}
  \end{flushright}
\end{frame}
}

% \section{Agenda}
% \begin{frame}
%   \frametitle{Agenda}
%   \begin{itemize}
%     \item 自己紹介
%     \item Agenda
%       \item Challenges/Experiences
%       \item Overcoming obstacles
%       \item Onboarding newcomers
%       \item Cultural Challenges
%     \item まとめ
%   \end{itemize}
% \end{frame}

\section{Introduction}
\begin{frame}
  \frametitle{Disclaimer}
  \begin{itemize}
    \item 本資料は、個人の見解です。
    \item 所属する企業・団体を代表する意見ではありません。
  \end{itemize}
\end{frame}

\begin{frame}
  \frametitle{Who am I?}
  \begin{itemize}
    \item Company:SUSE/ノベル株式会社
      \begin{itemize}
        \item QE(Quality Engineering) Team
        \item[] (日本にいるのは私だけ)
        \item \href{https://www.suse.com/newsroom/post/2016/suse-acquires-openstack-iaas-and-cloud-foundry-paas-talent-and-technology-assets-from-hpe-to-accelerate-growth-and-entry-into-new-markets/}{SUSE Acquires OpenStack IaaS and Cloud Foundry PaaS Talent and Technology Assets from HPE to Accelerate Growth and Entry into New Markets}
      \end{itemize}
    \item Job: Senior Software Engineer/Open Source Programmer
      \begin{itemize}
        \item \href{https://www.openstack.org/}{OpenStack}
         \href{https://wiki.openstack.org/wiki/QA}{QA} Upstream development, Core Reviewer
        \item[] (\href{https://docs.openstack.org/developer/tempest/}{Tempest},
         \href{http://status.openstack.org/openstack-health/}{OpenStack-Health},
         \href{https://docs.openstack.org/developer/subunit2sql/}{Subunit2SQL},
         \href{https://docs.openstack.org/developer/stackviz/}{Stackviz})
        \item \href{https://www.openstack.org/}{OpenStack}
        \item \href{http://stackalytics.com/?user_id=igawa&release=all&metric=all}{stackalytics.com/?user\_id=igawa}
      \end{itemize}
  \end{itemize}
\end{frame}

\subsection{Culural Challenges}
\begin{frame}
  \bf\Huge{Culural Challenges}
\end{frame}

\begin{frame}
\frametitle{Japanese}
  \begin{itemize}
    \item NOT to say "Yes / No" clearly
    \item Tend to be perfect
    \item Keep intonation
    \item Size of Economy
    \item Focusing on Reading and Writing
    \item Pronunciation and grammar are very different
      \begin{itemize}
      \item Pronouncing “L” vs “R” in words
      \item Subject-Verb-Object (E) vs Subject-Object-Verb (J)
      \end{itemize}
    \item Katakana: “ネットワーク” = “Network” (English)
    \item Kanji(漢字), Hiragana(ひらがな), Katakana(カタカナ)
    \item Network(ネットワーク), File(ファイル), Comment(コメント), etc..
    \item 和製英語(Japanese-made English): Paso-con(パソコン), Air-con(エアコン), Auto-bi(オートバイ)
      \begin{itemize}
        \item https://en.wikipedia.org/wiki/Wasei-eigo
      \end{itemize}
  \end{itemize}
\end{frame}

\begin{frame}
\frametitle{Chinese}
  \begin{itemize}
  \item Confucian culture
  \item Doctrine of the Mean -  one guideline is Leniency
  \item Like to say yes, don't like to say no
  \item Like to listen, don't like to negotiate
  \item Chinese pronunciation is not understood by others
  \item Not follow well with the English grammar
  \item Writing is hard because of the grammar but it can be understood
  \end{itemize}
\end{frame}

\begin{frame}
\frametitle{Brazilian}
  \begin{itemize}
  \item Conversations driven in similar way
  \item Short/direct responses may sound rude
  \item ``i'' (Portuguese) is pronounced as ``e'' (English)
  \item Grammar: adjectives position
  \item Some phonemes do not exist in Portuguese, e.g ``th'' vs ``f''
  \item Regular schools do a poor job teaching English
  \end{itemize}
\end{frame}


\subsection{Language Challenges}
\begin{frame}
  \bf\Huge{Language Challenges}
\end{frame}

\begin{frame}
\frametitle{Reading}
  \begin{itemize}
  \item Easiest in most of the time
  \item One of the most important
  \item IRC conversation goes fast
  \item Long emails, conclusion is unclear
  \end{itemize}
\end{frame}

\begin{frame}
\frametitle{Writing}
  \begin{itemize}
  \item Grammar
  \item Writing long and beautiful sentences is difficult
  \item Simpler sentences are prevalent
  \item Speed in IRC/chat
  \end{itemize}
\end{frame}

\begin{frame}
\frametitle{Listening}
  \begin{itemize}
  \item Variety of accents
  \item Speed
  \item Vocabulary
  \item Grammar
  \item Noisy environments
  \end{itemize}
\end{frame}

\begin{frame}
\frametitle{Speaking}
  \begin{itemize}
  \item Vocabulary
  \item Grammar
  \item Pronunciation
  \item Speed \& Fluency
  \end{itemize}
\end{frame}

\subsection{Overcoming Obstacles}
\begin{frame}
  \bf\Huge{Overcoming Obstacles}
\end{frame}

\begin{frame}
\frametitle{Overcoming Obstacles}
  \begin{itemize}
  \item Cultural challenges harder than language challenges
  \item Language immersion
  \item Forget limitations
  \item Do your best and you will eventually improve
  \item Reading gathers vocabulary
  \item Communicate daily
  \item Useful tools out there
  \item Practice with others or yourself
  \item One-to-one conversations
  \end{itemize}
\end{frame}


\subsection{Onboarding newcomers}
\begin{frame}
  \bf\Huge{Onboarding newcomers}
\end{frame}

\begin{frame}
\frametitle{Newcomers}
  \begin{itemize}
  \item Be friendly
  \item Find a mentor
  \item Share your opinion
  \item Prepare in advance
  \item Ask questions
  \item Brush up your English skills
  \end{itemize}
\end{frame}

\begin{frame}
\frametitle{Native Speakers}
  \begin{itemize}
  \item Be patient
  \item Speak slowly, please
  \item Use simple words and sentences
  \item Encourage communication
  \item Do not make fun
  \end{itemize}
\end{frame}


\subsection{Questions?}
\begin{frame}
  \frametitle{Questions?}
  \begin{itemize}
  \item[] More Informations
  \item \url{https://opensource.com/article/17/1/non-native-speakers-take-open-source-communities}
  \item \url{https://docs.openstack.org/contributor-guide/non-native-english-speakers.html}
  \end{itemize}
  \begin{center}
    \includegraphics[width=0.6\textwidth]{world_hands_diversity.png}
  \end{center}
  \begin{flushright}
    \bf\tiny{Image by}\tiny\url{ : opensource.com}
  \end{flushright}
\end{frame}

\end{document}
